a small commandline tool to convert Patterson files as generated by Ran\+Gen (see \href{http://www.projectmanagement.ugent.be/?q=research/data/RanGen}{\tt http\+://www.\+projectmanagement.\+ugent.\+be/?q=research/data/\+Ran\+Gen}) to a custom format while enabling the opportunity to output also a Graph\+M\+L (see \href{http://graphml.graphdrawing.org/}{\tt http\+://graphml.\+graphdrawing.\+org/}) representation to visualize the network i.\+e. with Ge\+Phi (\href{https://gephi.github.io/}{\tt https\+://gephi.\+github.\+io/}). Additionally, release and deadlines are generated to provide problem instances for scheduling problems with variable intensity and resource constraints.

\subsection*{Usage }

Ran\+Gen\+Conv needs a file in Patterson format (see \href{http://www.p2engine.com/p2reader/patterson_format}{\tt http\+://www.\+p2engine.\+com/p2reader/patterson\+\_\+format}) and will output a custom data file format (optional also a graphml file).


\begin{DoxyEnumerate}
\item type Ran\+Gen\+Conv -\/h to display usage \begin{quote}
Ran\+Gen\+Conv -\/h \end{quote}

\item to process a Patterson file \textquotesingle{}sample.\+rcp\textquotesingle{}, ignore the dummynodes at the start/end and save data to sample.\+dat type \begin{quote}
Ran\+Gen\+Conv sample.\+rcp sample.\+dat \end{quote}

\item to output with dummynodes use -\/d \begin{quote}
Ran\+Gen\+Conv -\/d sample.\+rcp sample.\+dat \end{quote}

\item to generate additionally a graphml file add -\/g. Note that Ran\+Gen\+Conv will automatically output the graphml to sample.\+dat.\+graphml \begin{quote}
Ran\+Gen\+Conv -\/d -\/g sample.\+rcp sample.\+dat \end{quote}

\item to alter the default time limit of 10 for the makewindow for an individual activity use option -\/t \begin{quote}
Ran\+Gen\+Conv -\/t 20 sample.\+rcp sample.\+dat \end{quote}

\item to check if a given file follows the Patterson format use -\/c \begin{quote}
Ran\+Gen\+Conv -\/c checkthisfile.\+rcp \end{quote}

\end{DoxyEnumerate}

\subsection*{Output file format }

Below is an example output file provided for the given network cf. \href{http://www.p2engine.com/p2reader/patterson_format}{\tt http\+://www.\+p2engine.\+com/p2reader/patterson\+\_\+format}. \begin{quote}
time = \{1,2,3,4,5,6,7,8,9,10,11,12,13,14,15,...,50,51,52,53,54\};

activity = \{1,2,3,4,5,6,7,8,9,10,11,12\};

resource = \{1,2,3,4\};

res\+\_\+capacity = \mbox{[}\mbox{[}10,20,8,10\mbox{]},\mbox{[}10,20,8,10\mbox{]},...,\mbox{[}10,20,8,10\mbox{]}\mbox{]};

max\+Progress = \mbox{[}0.\+166668,0.\+200001,0.\+333334,1,0.\+333334,0.\+500001,1,0.\+250001,0.\+333334,1,0.\+333334,0.\+200001\mbox{]};

min\+Progress = \mbox{[}0,0,0,0,0,0,0,0,0,0,0,0\mbox{]};

Relations = \{$<$1,9$>$,$<$2,5$>$,$<$2,6$>$,$<$2,7$>$,$<$3,8$>$,$<$4,10$>$,$<$5,12$>$,$<$6,11$>$,$<$7,13$>$,$<$9,12$>$,$<$10,12$>$,$<$11,13$>$\};

release = \mbox{[}1,3,1,9,9,9,4,9,10,11,14,19\mbox{]};

deadline = \mbox{[}8,12,5,28,21,20,10,24,33,23,39,49\mbox{]};

res\+\_\+demand = \mbox{[}\mbox{[}7,15,2,6\mbox{]},\mbox{[}1,8,4,8\mbox{]},\mbox{[}5,8,3,3\mbox{]},\mbox{[}6,15,2,6\mbox{]},\mbox{[}1,13,0,3\mbox{]},\mbox{[}2,16,2,0\mbox{]},\mbox{[}2,9,4,4\mbox{]},\mbox{[}8,12,5,5\mbox{]},\mbox{[}6,17,5,0\mbox{]},\mbox{[}2,10,2,5\mbox{]},\mbox{[}6,5,5,4\mbox{]},\mbox{[}8,10,3,7\mbox{]}\mbox{]}; \end{quote}


\subsection*{Documentation }

to generate a documentation of the source code, install doxygen and run \begin{quote}
doxygen doxy.\+doxy \end{quote}
via terminal. 